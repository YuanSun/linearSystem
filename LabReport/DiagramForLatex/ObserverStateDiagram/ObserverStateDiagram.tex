\documentclass[tikz,14pt]{standalone}
\usepackage{textcomp}
\usetikzlibrary{shapes,arrows}
\begin{document}
% Definition of blocks:
\tikzset{%
  block/.style    = {draw, thick, rectangle, minimum height = 2em,
    minimum width = 2em},
  sum/.style      = {draw, circle, node distance = 2cm}, % Adder
  input/.style    = {coordinate}, % Input
  output/.style   = {coordinate} % Output
}
% Defining string as labels of certain blocks.
\newcommand{\suma}{\Large$+$}
\newcommand{\inte}{$\displaystyle \int$}
\newcommand{\derv}{\huge$\frac{d}{dt}$}

\begin{tikzpicture}[auto, thick, node distance=1.5cm, >=triangle 45]
\draw
	% Drawing the blocks of observable canonical form
	% Forward parth
	node at (0,0)[right=-3mm]{\Large \textopenbullet}
	node [input, name=input1] {} 
	node [block, right of = input1](b0){$b_0$}
%	node [sum, right of=b0, node distance = 1.5cm] (suma1) {\suma}
	node [input, right of = b0, node distance = 1cm](x4dot){}
	node [block, right of=x4dot, node distance = 0.5cm] (inte1) {\inte}
	node [input, right of = inte1, node distance = 1cm](x4){}
	node [sum, right of = x4, node distance = 0.5cm] (suma2) {\suma}
	node [input, right of = suma2, node distance = 1cm](x3dot){}
	node [block, right of=x3dot, node distance = 0.5cm] (inte2) {\inte}
	node [input, right of = inte2, node distance = 1cm](x3){}
	node [sum, right of = x3, node distance = 0.5cm] (suma3) {\suma}
	node [input, right of = suma3, node distance = 1cm] (x2dot){}
	node [block, right of = x2dot, node distance = 0.5cm](inte3){\inte}
	node [input, right of = inte3, node distance = 1cm] (x2){}
	node [sum, right of = x2, node distance = 0.5cm] (suma4){\suma}
	node [input, right of = suma4, node distance = 1cm] (x1dot){}
	node [block, right of = x1dot, node distance = 0.5cm](inte4){\inte}
	node [input, right of = inte4, node distance = 1cm] (x1){}
	node [output, right of = x1, node distance = 0.5cm] (beforeout){}
	node [output, right of = beforeout, node distance = 0.5cm] (output){$\theta_2$}
	% Backward path
        node [block, below of=suma4] (a3) {$-a_3$}
        node [input, below of = a3](a3down){}
        node [block, below of=suma3] (a2) {$-a_2$}
        node [input, below of = a2](a2down){}
        node [block, below of= suma2] (a1) {$-a_1$}
        node [input, below of = a1](a1down){};
    % Joining blocks. 
    % Commands \draw with options like [->] must be written individually
    %Forward path
	\draw[->](input1) -- node {$T_D$}(b0);
 	\draw[-](b0) -- node {$\dot{x_4}$}(x4dot);
	\draw[->]node{}(x4dot) -- node{}(inte1);
	\draw[->](inte1) -- node{$x_4$}(x4) -- node{} (suma2);
	\draw[->](suma2) -- node{$\dot{x_3}$}(x3dot) -- node{}(inte2);
	\draw[->](inte2) -- node{$x_3$}(x3) -- node{} (suma3); 	
	\draw[->](suma3) -- node{$\dot{x_2}$}(x2dot) -- node{}(inte3);
	\draw[->](inte3) -- node{$x_2$}(x2) -- node{} (suma4);
	\draw[->](suma4) -- node{$\dot{x_1}$}(x1dot) -- node{}(inte4);
	\draw[->](inte4) -- node{$x_1$}(x1) -- node{}(beforeout) -- node{$\theta_2$} (output);	
	%Backward path
	% a3 path
	\draw[->](beforeout) |- node{} (a1down);
	\draw[->](a3down) -- node{}(a3);
	\draw[->](a3) -- node{}(suma4);
	% a2 path
	\draw[->](a3down) -- node{}(a1down);
	\draw[->](a2down) -- node{}(a2);
	\draw[->](a2) -- node{}(suma3);
	% a1 path
	\draw[->](a2down) -- node{}(a1down);
	\draw[->](a1down) -- node{}(a1);
	\draw[->](a1) -- node{}(suma2);

\end{tikzpicture}
\end{document}
