\documentclass[tikz,14pt]{standalone}
\usepackage{textcomp}
\usetikzlibrary{shapes,arrows}
\begin{document}
% Definition of blocks:
\tikzset{%
  block/.style    = {draw, thick, rectangle, minimum height = 2em,
    minimum width = 2em},
  sum/.style      = {draw, circle, node distance = 2cm}, % Adder
  input/.style    = {coordinate}, % Input
  output/.style   = {coordinate} % Output
}
% Defining string as labels of certain blocks.
\newcommand{\suma}{\Large$+$}
\newcommand{\inte}{$\displaystyle \int$}
\newcommand{\derv}{\huge$\frac{d}{dt}$}

\begin{tikzpicture}[auto, thick, node distance=1.5cm, >=triangle 45]
\draw
	% Drawing the blocks of observable canonical form
	% Forward parth
	node at (0,0)[right=-3mm]{\Large \textopenbullet}
	node [input, name=input1] {} 
	node [block, right of = input1](beforeZ){$\frac{1}{D_3(s)}$}
	node [input, right of = beforeZ, node distance = 1.5cm](Z){}
	node [block, right of= Z, node distance = 1cm] (afterZ) {$\frac{k}{gr\cdot J_d^* \cdot J_l}$}
	node [output, right of = afterZ, node distance = 1.5cm] (output){$\theta_2 (s)$};

    % Joining blocks. 
    % Commands \draw with options like [->] must be written individually
    %Forward path
	\draw[->](input1) -- node {$T_D$}(beforeZ);
 	\draw[->](beforeZ) -- node {$Z(s)$} (Z);
	\draw[->](Z) -- node{}(afterZ) -- node{$\theta_2(s)$}(output);
	\end{tikzpicture}
\end{document}
